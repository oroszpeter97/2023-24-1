\documentclass[]{article}

%opening
\title{}
\author{}

\begin{document}
\section{Kommutativitás}
	\subsection*{Szorzás:}
		A szorzás kommutatív, vagyis a két szám sorrendje nem számít. Például: $a * b = b * a$.
	\subsection*{Inner Join}
		Az "inner join" is kommutatív, vagyis a táblázatok sorrendje nem számít. Az eredmény ugyanaz lesz, függetlenül attól, hogy melyik táblázatot helyezzük az "inner join" művelet bal vagy jobb oldalára.

\section{Asszociativitás}
	\subsection{Szorzás:}
		A szorzás is asszociatív, vagyis a műveletek sorrendjének megváltoztatása nem változtatja meg az eredményt. Például: $(a * b) * c = a * (b * c)$.
	\subsection{Inner Join:}
		Az "inner join" is asszociatív, azaz a sorrend, amelyben több "inner join" műveletet alkalmazunk, nem változtatja meg az eredményt. Például:\\ (A INNER JOIN B) INNER JOIN C = A INNER JOIN (B INNER JOIN C).
\end{document}
